\documentclass[12pt, a4paper]{article}

% VARIABLES
% Change these to your essay.
\def\thetitle{Parallel Programming Languages and Systems Exercise 1}
\def\theauthor{Christopher Brown -- 0824586}

% DOCUMENT LAYOUT
\usepackage{geometry}
% Change this if you have specific margins to use.
% \geometry{a4paper, hmargin=0.8in}

% FONTS
\usepackage{fontspec}
\usepackage{xunicode}
\usepackage{xltxtra}
\defaultfontfeatures{Mapping=tex-text}
\setromanfont[Ligatures={Common},
    BoldFont={Adobe Garamond Pro Bold},
ItalicFont={Adobe Garamond Pro Italic}]{Adobe Garamond Pro}
\setmonofont[Scale=0.8]{Monaco}

% HEADERS
\usepackage{fancyhdr}
\pagestyle{fancyplain}
\rhead{0824586}
\lhead{Parallel Programming Languages and Systems}
\setlength{\headheight}{14.5pt}

% HEADINGS
\usepackage{sectsty}
\usepackage[normalem]{ulem}
\sectionfont{\rmfamily\mdseries\upshape\Large}
\subsectionfont{\rmfamily\bfseries\upshape\normalsize}
\subsubsectionfont{\rmfamily\mdseries\upshape\normalsize}

% SYMBOLS & MATHS
\usepackage{amsmath}
\usepackage{marvosym}
\usepackage{url}

%% SOURCE CODE
\usepackage{minted}

\renewcommand{\theFancyVerbLine}{\ttfamily \textcolor[rgb]{0.5,0.5,0.5}{\small {\arabic{FancyVerbLine}}}}

%% TODOs
%\usepackage{todonotes}
%\usepackage{cite}

% PDF SETUP
%\usepackage[dvipdfm,unicode,bookmarks,colorlinks,breaklinks,pdftitle={\thetitle},pdfauthor={\theauthor}]{hyperref}
%\hypersetup{linkcolor=black,citecolor=black,filecolor=black,urlcolor=blue}

% TITLE PAGE
\title{\normalsize{\textit{Parallel Programming Languages \& Systems}} \\ \Huge Exercise Sheet 1}
\author{\theauthor}

\begin{document}

\maketitle
\thispagestyle{empty}

\begin{enumerate}

    \item

        \begin{enumerate}

            \item

                Since there is only one ordering of these instructions then there can
                be only one final state in this case. The values in this case are:
                $x=5$, $y=2$.

            \item

                There are 6 different orderings of the instructions in this case:

                \begin{itemize}

                    \item S1, S2, S3
                    \item S1, S3, S2
                    \item S2, S1, S2
                    \item S2, S3, S1
                    \item S3, S1, S2
                    \item S3, S2, S1

                \end{itemize}

                The values of $(x,y)$ for each of these cases are respectively:
                ($x=5$, $y=2$), ($x=2$, $y=-3$), ($x=2$, $y=-3$), ($x=2$,
                $y=-3$), ($x=2$, $y=-3$), ($x=-11$, $y=-8$).

                Therefore there are 3 different possible final states.

            \item

                This code can never be successfully executed as it will never
                terminate. This is due to the fact that the \texttt{await} can
                never complete. The starting state of the system does not
                satisfy the requirement and the only possible thing it can
                execute instead is the S3 in the second block after which the
                \texttt{await} conditional is still not satisfied. Therefore
                this code will never terminate.

        \end{enumerate}

    \item

        There are 7 different execution possibilities for this code:
        2 which don't terminate and 5 which do. These can be loosely
        categorised into 4 different categories based on the previous
        sequence of events that led to the system being in that
        state.

        \begin{description}

            \item[First Block Bias] \hfill \\
                In order to get into a state that doesn't terminate the
                system can execute the entire while loop (the await
                cannot execute until the last \texttt{x=x-1} has been
                executed) and then execute \texttt{y=y+1} before the
                \texttt{await} can check again. This leads to the first
                block being finished and the second block stuck on the
                \texttt{await} because $x=0$ and $y=1$.

            \vspace{0.2cm}
            \item[Awake Executes] \hfill \\
                This scenario starts the same way as above except this
                time after the while loop is complete the
                \texttt{await} is executed. This frees up the rest of
                the second block for execution. Depending on the
                execution order of \texttt{y=2} and \texttt{y=y+1} we
                can either end up with ($x=8$, $y=2$) or ($x=8$,
                $y=3$).

            \vspace{0.2cm}
            \item[Interrupted While Loop] \hfill \\
                The next scenario involves the possibility that between
                the last \texttt{x=x+1} and the while loop checking
                that it is complete the \texttt{await} and the
                \texttt{x=8} of the second block execute meaning that
                the while loop continues.

                After the while loop has run down again and depending
                on how the final instructions of each block are
                executed there are 3 states that the system can end up
                in: ($x=2$, $y=3$), ($x=0$, $y=2$), or ($x=0$, $y=3$).

            \vspace{0.2cm}
            \item[Unlucky Y Assignment] \hfill \\
                There is another possibility that is similar to the
                scenario above. Instead of the while loop counting all
                the way down a second time when \texttt{x} reaches
                1 the final statement (\texttt{y=2}) is executed from
                the second block. Since the while loop condition will
                now never be satisfied - since integers do not over or
                underflow in this environment - the system will never
                terminate.

        \end{description}

    \item

        My pseudocode solution to the critical section problem is shown below.

        \begin{minted}[tabsize=2,linenos,gobble=12,xleftmargin=1cm]{c}
            int l = 1;
            co [i = 1 to n] {
            while(something) {
                while (l == 1 || DEC(l));
                critical section;
                l = 1;
                non-critical section;
              }
            }
        \end{minted}

        The first thread to reach line 4 will use \texttt{DEC} to decrement the
        lock variable (\texttt{l}) to 0 before entering the critical section
        due to the lock being $\geq 0$. The next thread to reach this stage
        (ignoring the \texttt{l == 1} for now) will decrement the lock variable
        again but in this case \texttt{DEC} will return true since the lock
        variable is $< 0$. This means that the thread will continue spinning
        indefinitely.

        In conventional systems once the lock variable underflowed when enough
        threads had attempted to access the critical section then the integer
        would take its largest possible value before allowing many threads to
        enter the critical section at once. However, since this is
        a theoretical system integers are defined as having limitless scale in
        both directions and this problem does not arise.

        In order to improve cache performance I have added an additional
        condition to the while loop condition on line 4. The expression
        \texttt{l == 1} means that the problematic \texttt{DEC} function is
        only called if there is a possibility that it may return false and
        allow for continued execution of the current thread.

\end{enumerate}

\end{document}
