\documentclass[12pt, a4paper]{article}

% VARIABLES
% Change these to your essay.
\def\thetitle{Dropcycle: A Cloud Infrastructure Problem}
\def\theauthor{Christopher Brown -- 0824586}

% DOCUMENT LAYOUT
\usepackage{geometry}
% Change this if you have specific margins to use.
% \geometry{a4paper, hmargin=0.8in}

% FONTS
\usepackage{fontspec}
\usepackage{xunicode}
\usepackage{xltxtra}
\defaultfontfeatures{Mapping=tex-text}
\setromanfont[Ligatures={Common},
	BoldFont={Adobe Garamond Pro Bold},
	ItalicFont={Adobe Garamond Pro Italic}]{Adobe Garamond Pro}
\setmonofont[Scale=0.8]{Monaco}

% HEADERS
\usepackage{fancyhdr}
\pagestyle{fancyplain}
\rhead{0824586}
\lhead{\textsc{\thetitle}}
\setlength{\headheight}{14.5pt}

% HEADINGS
\usepackage{sectsty}
\usepackage[normalem]{ulem}
%\sectionfont{\rmfamily\mdseries\upshape\Large}
%\subsectionfont{\rmfamily\bfseries\upshape\normalsize}
%\subsubsectionfont{\rmfamily\mdseries\upshape\normalsize}

% SYMBOLS & MATHS
\usepackage{amsmath}
\usepackage{marvosym}
\usepackage{url}

% SOURCE CODE
\usepackage{listings}

% TODOs
\usepackage{todonotes}

% PDF SETUP
\usepackage[dvipdfm,unicode,bookmarks,colorlinks,breaklinks,pdftitle={\thetitle},pdfauthor={\theauthor}]{hyperref}
\hypersetup{linkcolor=black,citecolor=black,filecolor=black,urlcolor=blue}

% TITLE PAGE
\title{\thetitle}
\author{\theauthor}
\date{\today}

\begin{document}

\maketitle
\thispagestyle{empty}

\section{Introduction}

This essay considers the challenges and problems of the startup
\emph{Dropcycle} and how they can utilise cloud computing to solve these
problems in a scalable, effective and cost-efficient manner. The main problem
that they are trying to solve is the task of migrating processes that are being
starved of resources from a users computer to a more powerful server in another
location. They have a number of requirements that add additional challenges to
the already challenging problem.

\section{Design Considerations}

\emph{Dropcycle} have a number of functional and non-functional requirements
that are both specified and unspecified. I will address each of these
throughout the next section and outline a solution to each of them. For ease of
description I will also assume that the Infrastructure-as-a-service provider
has similar capabilities to \emph{Amazon Web Services}.

\subsection{Functional Requirements}

The main and specified functional requirement is the migration of starved
processes from the users local machine to a more powerful server in another
location. We will assume that the user is able to set limits for the resources
that they want the process to consume and that the first problem we must solve
in the work flow is to migrate the process to the remote machine.

Once a process has been marked as being starved of resources then the first
thing that \emph{Dropcycle} should do is stop it to allow for an easy
migration. Trying to migrate a live running process is likely to be an exercise
in futility as the process will change before the new state can be propagated
to the server. (Besides, the migration of a live running process would be
a good headline feature for v2.0!). Once the process has been stopped with
a signal or similar then the next task is to serialise all of the data related
to that process that would allow it to be reconstructed and restarted again on
the server. A kernel module in the case of Linux and Macintosh or a driver on
Windows sits low enough in the operating system that it would be able to get
this information and pass it to the \emph{Dropcycle} application. This data
would include the program counter, register state and virtual memory image.

Once this has been completed then the process state information can be
serialised and sent to the server that that will

different types of process task storage/cpu/memory
uptime
recovery
multi os support

\section{Solutions}

\section{Proposed Solution}

\section{Conclusion}

\end{document}
