\documentclass[12pt, a4paper]{article}

% VARIABLES
% Change these to your essay.
\def\thetitle{Dropcycle: A Cloud Infrastructure Problem}
\def\theauthor{Christopher Brown -- 0824586}

% DOCUMENT LAYOUT
\usepackage{geometry}
% Change this if you have specific margins to use.
% \geometry{a4paper, hmargin=0.8in}

% FONTS
\usepackage{fontspec}
\usepackage{xunicode}
\usepackage{xltxtra}
\defaultfontfeatures{Mapping=tex-text}
\setromanfont[Ligatures={Common},
	BoldFont={Adobe Garamond Pro Bold},
	ItalicFont={Adobe Garamond Pro Italic}]{Adobe Garamond Pro}
\setmonofont[Scale=0.8]{Monaco}

% HEADERS
\usepackage{fancyhdr}
\pagestyle{fancyplain}
\rhead{0824586}
\lhead{\textsc{\thetitle}}
\setlength{\headheight}{14.5pt}

% HEADINGS
\usepackage{sectsty}
\usepackage[normalem]{ulem}
%\sectionfont{\rmfamily\mdseries\upshape\Large}
%\subsectionfont{\rmfamily\bfseries\upshape\normalsize}
%\subsubsectionfont{\rmfamily\mdseries\upshape\normalsize}

% SYMBOLS & MATHS
\usepackage{amsmath}
\usepackage{marvosym}
\usepackage{url}

% SOURCE CODE
\usepackage{listings}

% TODOs
\usepackage{todonotes}

% PDF SETUP
\usepackage[dvipdfm,unicode,bookmarks,colorlinks,breaklinks,pdftitle={\thetitle},pdfauthor={\theauthor}]{hyperref}
\hypersetup{linkcolor=black,citecolor=black,filecolor=black,urlcolor=blue}

% TITLE PAGE
\title{\thetitle}
\author{\theauthor}
\date{\today}

\begin{document}

\maketitle
\thispagestyle{empty}

\section{Introduction}

\emph{Dropcycle} is a startup that is planning on using cloud and distributed
computing to solve the problem of migrating and then running processes from
a user's machine to a more powerful system. They are aiming to achieve this
goal in a cost-effective, scalable and effective manner. In this essay the
author outlines use cases of the service, possible solutions to the overall
problem with their advantages and disadvantages before outlining the solution
that would best fit the task at hand.

\section{Use Cases}

The basic premise of \emph{Dropcycle} is to allow the user to select processes
to be monitored for excessive use of the user's system. When one of these
processes surpass a calculated or predefined limit then the offending process
should be migrated to run on a powerful \emph{Dropcycle} server instead. The
use cases for a service such as this are broad in complexity, scale, and type
of task. The service should be able to handle as many of these as is
commercially and functionally viable.

\subsection{Client-side Application}

The flow through the system always begins with an application running on the
user's system. The application has a number of important responsibilities in
the migration process. The first of these is monitoring current processes to
make sure that they are not consuming more resources than the system can handle
and therefore starve other processes while crippling themselves. The
application must monitor a number of attributes of the watched processes to
achieve this. The three main attributes are the CPU usage, main memory usage,
and secondary storage usage.

Once the client has detected an overrun in once of these areas then it must
begin it's second responsibility: carrying out an action to remove the load on
the system so that it can return to normal operation. Most of the time this
will involve preparing the offending process for migration. However, depending
on the implementation, in some cases it may be possible to meet this
responsibility without making the migration. For example, a lack of secondary
storage may not require the process to migrate due to the ability of some
implementations (to be discussed later) that could add secondary storage space
to a running system.

Due to the need to make sure that the process is migrated to the best suited
machine it is also important for the client to monitor other process attributes
that may allow for a better understanding of the requirements of the process.
The client could monitor attributes such as the number of threads and how the
program is using them. It could monitor the types of files that are open and
how the process is using them. For example, a movie encoder is a process that
uses a considerable amount of secondary storage and it will probably have one
file it is writing to that has a recognisable video file extension.
\emph{Dropcycle} could use this information to start provisioning more hard
drive space for the user in the cloud before it is needed to improve the
customer experience.

Depending on the implementation details of the server-side the client may also
be responsible for other tasks that enable the entire service to function. This
may involve managing hooks into the user's kernel to help with the sharing of
memory, mounting of file systems, and helping with inter-process communication.
Unfortunately, the presence of these responsibilities rely wholly upon the
implementation of the service and as such this cannot be discussed until after
the service solution has been defined.

The final responsibility of the client is, rather regrettably, boring from
a technical standpoint. The user should be able to set per-process limits on
the resources that the process can use. This information should then be passed
to the \emph{Dropcycle} servers so that they may enforce these limits.

The client has a number of business requirements that affect it's usage on
different user systems. Due to the requirement that \emph{Dropcycle} must run
on the 3 market dominant operating systems of today\footnote{Windows, Mac, and
Linux} any functionality that is implemented on one of the platforms must be
ported and possible to implement on the others. Additionally, due to the
extreme portability of the Linux kernel and its possible inclusion in the
supported platforms for \emph{Dropcycle}, the application and service should
take into account the possibility of a wide selection of processor
architectures to support.

\subsection{Server-side Infrastructure \& Software}

The server side of the service has a number of all technical responsibilities
that it must handle. However, as with some of the client responsibilities, not
all of them will be required depending on the implementation of the system.

The first and most obvious of these responsibilities is that the server should
be able to receive a process from the client application and complete the
migration so that the process is running again. This has the imposed
restriction and requirement that the servers must all be running the same
version of Linux yet they are required to receive a migrated process from
3 different operating systems spanning countless architectures.

Imposing limits on the resources a process can use is going to be another
responsibility of the server side. The user is limited by their account and by
any additional limits they have placed on a process. This will involve
monitoring the usage of memory and storage space in some way and imposing hard
and soft limits to the migrated process. The monitoring of CPU usage is largely
dependent on solution and business decisions. Do we limit the clock speed,
total number of cycles, both, or another performance metric?

In a similar note, the server will be responsible for the billing component of
the whole system. \emph{Dropcycle} will need to not just limit the resources
being used but also keep metrics and logs of what resources were being used in
order to send people their bill at the end of each month.

%TODO: More server responsibilities

\section{Solutions}

\subsection{Client Application Solutions}

\section{Proposed Solution}

\section{Conclusion}

\end{document}
